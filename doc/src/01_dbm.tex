% Section 1: DBM Theory and Usage
\section{Difference Bound Matrices (DBM)}
\subsection{Mathematical Foundations}
A \emph{zone} over a finite set of clocks \(C = \{x_1,\dots,x_n\}\) is a set of clock valuations satisfying conjunctions of constraints of the form:
\[ x_i - x_j \ \triangleleft\ c \quad (0 \le i,j \le n,\; c \in \mathbb{Z},\; \triangleleft \in \{<,\le\}) \]
with the special clock \(x_0\) fixed to 0. Zones are convex polyhedra in \(\mathbb{R}_{\ge 0}^n\).

A Difference Bound Matrix (DBM) is an \((n+1)\times(n+1)\) matrix \(M\) where each entry \(M[i,j]\) encodes a constraint \(x_i - x_j \triangleleft c\). The canonical form (a closed DBM) ensures minimal bounds by computing the all-pairs shortest paths closure (Floyd–Warshall variant). The empty zone is detected if any diagonal entry violates \(x_i - x_i \le 0\).

\subsection{Core Operations}
Let \(D\) denote a DBM:
\begin{itemize}
  \item \textbf{Initialization} (Zero zone): All clocks equal 0: \(x_i = 0\). DBM: \(M[i,0] = (0,\le)\), \(M[0,i] = (0,\le)\).
  \item \textbf{Up} (Time elapse): Removes upper bounds that prevent uniform time increase while preserving differences: formally \(\mathrm{Up}(Z) = \{\, v + d \mid v\in Z,\; d\in \mathbb{R}_{\ge 0}\,\}\) restricted by invariants.
  \item \textbf{Guard intersection}: Add constraint \(x_i - x_j \triangleleft c\) by tightening \(M[i,j]\).
  \item \textbf{Reset} of clock \(x_k\): Replace constraints involving \(x_k\) by those implied from setting \(x_k := 0\) (i.e., \(x_k\) becomes aligned with reference \(x_0\) after closure and guard application).
  \item \textbf{Invariants}: Intersect zone with location invariant constraints before allowing time to pass or after transitions.
  \item \textbf{Extrapolation}: Abstract zone by widening bounds exceeding maximal constants (LU / Max-bounds) to ensure finiteness of the symbolic state space.
\end{itemize}

\subsection{Relations Between DBMs}
Given two canonical DBMs \(D_1\) and \(D_2\) over same dimension:
\begin{itemize}
  \item Inclusion: \(D_1 \subseteq D_2\) iff for all \(i,j\): \(bound(D_1[i,j]) \le bound(D_2[i,j])\).
  \item Equality: Mutual inclusion.
  \item Emptiness: Diagonal \(M[i,i]\) strictly negative (encoded sentinel) after closure.
\end{itemize}

\subsection{Closure Algorithm}
Closure computes minimal constraints: for all \(i,j,k\), update \(M[i,j] = \min(M[i,j], M[i,k] + M[k,j])\). Complexity: \(O(n^3)\), with \(n = |C|+1\). Needed after each batch of constraint tightenings (guards, resets) to maintain canonical form.

\subsection{Finiteness via Extrapolation}
For timed automata with maximal constant K (from guards/invariants), zone graph may still explode. Extrapolation enforces an abstraction: if an upper-bound exceeds K, replace it by \(\infty\); if a lower-bound is below -K adjust to -K. LU-extrapolation refines this using per-clock separate lower (L) and upper (U) bounds.

\subsection{Implementation Touchpoints}
In code, helper calls include:
\begin{lstlisting}
dbm_init(zone.data(), dim);       // zero zone
if(!dbm_close(zone.data(), dim))  // closure / emptiness
    ...
dbm_up(zone.data(), dim);         // Up operation
dbm_constrain1(zone.data(), dim, i, j, value); // add guard
// resets handled in apply_transition (not shown here)
dbm_extrapolateMaxBounds(zone.data(), dim, U.data()); // widening
\end{lstlisting}
All operations act in-place on the contiguous matrix (\texttt{std::vector\textless raw\_t\textgreater}). Canonical pointers are not stored; structural sharing is avoided for simplicity.

\subsection{Worked Example (2 Clocks)}
Consider clocks $x$ and $y$ with reference $x_0=0$. Start in the zero zone (both clocks 0). We apply:
\begin{enumerate}
  \item Guard $x \le 5$.
  \item Guard $y - x < 3$.
  \item Invariant at location: $y \le 7$.
  \item Time elapse (Up).
  \item Reset $x$ (e.g. transition resets $x$).
  \item Extrapolation with $U(x)=U(y)=5$.
\end{enumerate}

We encode each DBM entry as $(c,\prec)$ where $\prec \in \{<,\le\}$. Initial (zero) DBM (row index $i$, column $j$ encodes $x_i - x_j \prec c$):
\begin{center}
\begin{tabular}{c|ccc}
 & $x_0$ & $x$ & $y$ \\\hline
$x_0$ & $(0,\le)$ & $(0,\le)$ & $(0,\le)$ \\
$x$   & $(0,\le)$ & $(0,\le)$ & $(0,\le)$ \\
$y$   & $(0,\le)$ & $(0,\le)$ & $(0,\le)$ \\
\end{tabular}
\end{center}

After applying guard $x \le 5$: tighten entry for $x - x_0$ giving $(5,\le)$. Guard $y - x < 3$ tightens $M[y,x]$ to $(3,<)$. Closure propagates $y - x_0 < 8$ but invariant $y \le 7$ later tightens to $(7,\le)$. Time elapse (Up) removes upper bounds from the reference to allow delay: entries of form $x_i - x_0 \prec c$ with $i>0$ may widen to $(\infty,\le)$ except those restricted by invariants (so $y$ retains $\le 7$). Reset $x$ sets $x=0$: copy row/column of $x_0$ except constraints involving other clocks; closure re-tightens $y - x < 3$.

Final (simplified) canonical constraints before extrapolation (assuming invariant applied):
\[
  x \le 5,\quad y \le 7,\quad y - x < 3,\quad x \ge 0,\; y \ge 0.
\]
Extrapolation with upper bound 5 widens $y \le 7$ to $y < \infty$ (since $7>5$) but keeps $x \le 5$. Thus we obtain an abstracted zone ensuring finiteness.

\paragraph{Code Trace.} The sequence in code:
\begin{lstlisting}[language=C++]
dbm_init(dbm.data(), dim);          // zero
dbm_constrain1(dbm.data(), dim, X, 0, pack_le(5));      // x <= 5
dbm_constrain1(dbm.data(), dim, Y, X, pack_lt(3));      // y - x < 3
dbm_close(dbm.data(), dim);                         // closure
apply_invariant(dbm, loc_inv);                      // y <= 7
dbm_up(dbm.data(), dim);                            // Up
reset_clock(dbm, X);                                // x := 0
dbm_close(dbm.data(), dim);
dbm_extrapolateMaxBounds(dbm.data(), dim, U.data());    // extrapolate
\end{lstlisting}
This mirrors the theoretical progression: guard intersection, closure, invariant, Up, reset, closure, extrapolation.

\paragraph{Emptiness Corner Case.} If we also added guard $y - x < 0$ after $y - x < 3$, closure would derive $y - y < 0$ (diagonal negative) and report emptiness.
